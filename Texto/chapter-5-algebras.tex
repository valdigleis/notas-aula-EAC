% Teoria dos conjuntos
\chapter{Introdução à Álgebra Universal}\label{cap:Semigroups}

\epigraph{``Na matemática, tudo são funções!''}{Autoria desconhecida.}

\section{$\Sigma$-Álgebras}\label{sec:basicSemigroups}

A definição de álgebra que será apresentada mais adiante captura a noção das mais diferentes estruturas algébricas já muito bem conhecidas (e também as menos conhecidas) na literatura, de fato, a definição que será dada mais adiante é tão forte que cunhou todo o novo campo de estudo dentro da matemática, chamado de Álgebra Universal \cite{denecke2018}. Um ponto de destaque que vale ser mencionado é que, diversas pesquisas recentes em lógica \cite{plotkin2012, riche2005}, teoria de funções recursivas \cite{rajesh2025}, teoria de autômatos \cite{gorrieri2023}, linguagens de programação e especificações formais, revelaram que todas essas áreas podem tirar de fato muito proveito da ideia de Álgebra Universal.

Embora a definição de álgebra universal seja algo relativamente recente\sidefootnote{Remotando aproximadamente meados dos anos 30 com Birkhoff \cite{burris1981}.}, vários matemáticos do final do século 19 já especulavam por algo próximo de nossas ideias modernas, de fato, Whitehead em 1898, e mais tarde por Noether, argumentaram sobre formalizações similares. Antes de formalizar o conceito de o que seria uma álgebra é necessário formalizar algumas ferramentas antes.

\begin{definicao}[$\Sigma$-assinatura]\label{def:SigmaAssinatura}
  $\Sigma$ é um conjunto (possivelmente infinito) de símbolos e $arid: \Sigma \rightarrow \mathbb{N}$ é uma função total, uma $\Sigma$-assinatura é uma estrutura $\langle \Sigma, arid \rangle$.
\end{definicao}

A função $arid$ retorna para todo $f \in \Sigma$ sua aridade, ou seja, a quantidade de argumentos que o símbolo $f$ necessita. Em outras palavras, a $\Sigma$-assinatura descreve a sintaxe da linguagem, pois, por exemplo, pegando um $f \in \Sigma$ tal que $arid(f) = 3$, fica implícito a escrita de palavras da forma $f(x, y, z)$ onde $x, y$ e $z$ são também palavras da linguagem.

\begin{definicao}[$\Sigma$-semântica]\label{def:SemanticaSigmaAlgebra}
  Seja $A$ um conjunto não vazio, dado uma $\Sigma$-assinatura $\langle \Sigma, arid \rangle$, a semântica dos símbolos funcionais\sidefootnote{Em especial quando $n = 0$ tem-se que $A^0 = \{\emptyset\}$, e assim todo $f: A^0 \rightarrow A$ pode ser vista como uma função $f(\emptyset) = a$ com $a \in A$.} em $A$ é um conjunto da forma:
  \begin{equation}
    \Sigma_A = \{f \in \Sigma \mid arid(f) = n \land f: A^n \rightarrow A \text{ é uma função total}\}
  \end{equation}
\end{definicao}

A Definição \ref{def:SemanticaSigmaAlgebra} apresenta a ideia de uma semântica associada de uma $\Sigma$-assinatura com respeito a um conjunto base $A$. Usando uma visão mecância pode-se ver esse conjunto base como sendo os dados usados por uma máquina e os símbolos funcionais são os programas compilados para serem executados pela máquina\sidefootnote{Do ponto de vista formal, pelo teorema da universalidade de Turing \cite{hopcroft2008} e pela tese de Chuch programas e máquinas são entendidas equivalentes \cite{benjaLivro2010}.}. E através dessa visão, podemos arguementar que nem todo símbolo funcional poderá ser interpretado no conjunto base, ou seja, nem todo programa poderá ser compilado  para ser executado pela máquina.

\begin{exemplo}\label{exe:Algebra1}
  Dado a $\Sigma$-assinatura  $\langle \{S, V\}, arid \rangle$ com $arid(S) = 1$ e $arid(V) = 2$. Agora para os conjuntos $\mathbb{N}$ e $\mathbb{Z}$ e $\mathbb{Q}$ tem-se as seguintes interpretações:
  \begin{itemize}
    \item[(a)] Para $\mathbb{N}$ tem-se $S(n) = n + 1$ e $V(m, n) = max(m, n)$.
    \item[(b)] Para $\mathbb{Z}$ tem-se $S(n) = 1 - n $ e $V(m, n) = -2m + 3n$.
    \item[(c)] Para $\mathbb{Q}$ tem-se $S\Big(\frac{m}{n}\Big) = m$ e $V$ não é interpretado em $\mathbb{Q}$.
  \end{itemize}
\end{exemplo}

Para algum conjunto $A$, os símbolos funcionais de aridade zero na $\Sigma$-assinatura podem ser visto apenas como apelidos para os elementos em $A$, de fato, isso é algo interessante do ponto de vista de linguagem, pois uma vez que, todo elemento em $A$ é visto como uma função, tem-se (nesta visão) que os funcionais $n$-ários são então funções de alta ordem em que seu argumentos não são meros elementos de $A$, mas sim funções sobre $A$, o que volta para nossa epígrafe inícial ``\textbf{Na matemática, tudo são funções!}''.

\begin{definicao}[$\Sigma$-álgebra]\label{def:SigmaAlgebra}
  Uma $\Sigma$-álgebra é uma estrutura $\langle A, \mathcal{F} \rangle$ onde $\mathcal{F} \subseteq \Sigma_A$ é um conjunto finito e não vazio.
\end{definicao}

Uma vez que $\mathcal{F} = \{f_1, \cdots, f_n\}$ geral é estabelecido a convenção que $arid(f_1) \geq \cdots \geq arid(f_n)$, assim é dito que o tipo da $\Sigma$-álgebra é exatamente a sequência $(arid(f_n))_{i \in \{1, \cdots, n\}}$. E a partir dessa convenção é possível estabelecer uma ordem parcial sobre a classe de todas as $\Sigma$-álgebras como discutido em \cite{denecke2018}.

\begin{exemplo}
  Considere as $\Sigma$-álgebras formadas pelas interpretações apresentadas no Exemplo \ref{exe:Algebra1}, ou seja, as $\Sigma$-álgebras $\langle \mathbb{N}, \{V, S\} \rangle$, $\langle \mathbb{Z}, \{V, S\} \rangle$ e $\langle \mathbb{Q}, \{S\} \rangle$ tem-se que a primeira e a segunda são do tipo $(2, 1)$ e a última é do tipo $(1)$.
\end{exemplo}

\begin{atencao}
  É comum como pode ser visto em \cite{burris1981, denecke2018, carmo2013} usar a notação sem as chaves do conjunto $\mathcal{F}$, ou seja, quando $\mathcal{F} = \{f_1, \cdots, f_n\}$ a escrita de uma $\Sigma$-álgebra é da forma $\langle A, f_1, \cdots, f_n \rangle$, em vez de, $\langle A, \{f_1, \cdots, f_n\} \rangle$.
\end{atencao}

